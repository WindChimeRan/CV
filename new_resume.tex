% !TEX program = pdflatex
%%%%%%%%%%%%%%%%%%%%%%%%%%%%%%%%%%%%%%%%%
% Medium Length Professional CV
% LaTeX Template
% Version 2.0 (8/5/13)
%
% This template has been downloaded from:
% http://www.LaTeXTemplates.com
%
% Original author:
% Rishi Shah 
%
% Important note:
% This template requires the resume.cls file to be in the same directory as the
% .tex file. The resume.cls file provides the resume style used for structuring the
% document.
%
%%%%%%%%%%%%%%%%%%%%%%%%%%%%%%%%%%%%%%%%%

%----------------------------------------------------------------------------------------
%	PACKAGES AND OTHER DOCUMENT CONFIGURATIONS
%----------------------------------------------------------------------------------------

\documentclass{resume} % Use the custom resume.cls style

\usepackage[left=0.75in,top=0.6in,right=0.75in,bottom=0.6in]{geometry} % Document margins
\usepackage{hyperref}
\newcommand{\tab}[1]{\hspace{.2667\textwidth}\rlap{#1}}
\newcommand{\itab}[1]{\hspace{0em}\rlap{#1}}
\name{Haoran Zhang} % Your name
\address{University of Illinois Urbana-Champaign, Illinois, US}
% \address{github.com/WindChimeRan} % Your address
%\address{123 Pleasant Lane \\ City, State 12345} % Your secondary addess (optional)
\address{github.com/WindChimeRan \\ haoranz6@illinois.edu} % Your phone number and email

\begin{document}

%----------------------------------------------------------------------------------------
%	EDUCATION SECTION
%----------------------------------------------------------------------------------------

\begin{rSection}{Education}

{\bf University of Illinois Urbana-Champaign} \hfill {\em August 2019 - Present} 
\\ Master, Information Management
\\{\bf Changsha University of Science \& Technology} \hfill {\em July 2014 - June 2018} 
\\ Bachelor, Computer Science
%Minor in Linguistics \smallskip \\
%Member of Eta Kappa Nu \\
%Member of Upsilon Pi Epsilon \\


\end{rSection}

\begin{rSection}{Research Experience}
    {\bf Changsha University of Science \& Technology} \hfill {\em May 2018 - Sep 2019} 
\\ Research Assistant on Relation Extraction \hfill {\em Supervised by Daojian Zeng}
\\{\bf University of California, Los Angeles} \hfill {\em Sep 2016 - June 2017} 
\\ Summer Research on Computer Vision \hfill {\em Supervised by Yajia Yang}
\end{rSection}
% \begin{rSection}{Carrier Objective}
%  To work for an organization which provides me the opportunity to improve my skills and knowledge to grow along with the organization objective.
% \end{rSection}
%--------------------------------------------------------------------------------
%    Projects And Seminars
%-----------------------------------------------------------------------------------------------
\begin{rSection}{Projects}
{\bf Open Source Relation Extraction}
\begin{enumerate}
    \item reproduction of papers: \href{https://github.com/WindChimeRan/pytorch_multi_head_selection_re}{\textit{Multihead Selection}} and \href{https://github.com/WindChimeRan/pytorch_copy_re}{\textit{CopyRE}} (50 stars)
    \item summaries and tags of papers: \href{https://github.com/WindChimeRan/NREPapers2018}{\textit{RE2018}} and \href{https://github.com/WindChimeRan/NREPapers2019}{\textit{RE2019}} (35 stars)
\end{enumerate}

% reproduction of papers \href{https://github.com/WindChimeRan/pytorch_multi_head_selection_re}{\textit{Multihead Selection}} and \href{https://github.com/WindChimeRan/pytorch_copy_re}{\textit{CopyRE}}
% % \\The project aims at designing and fabrication of two Buckling Restrained Braces which were analyzed under dynamic loading. As alternative for conventional braces, these BRBs are also beneficial for seismic retro-fitting in RCC and steel structures.\\
{\bf Distant Supervised Relation Extraction without False Negative Noise} (On going)\\
    NA relation covers 79\% bags in the prevalent DSRE dataset, where false negative noise plays an important role in the prediction. We discussed the causes of the false negative noise and put forward a reasonable assumption. Based on the assumptions and external fine-grained entity type information, our model works better on the pure test set.

{\bf Joint Extraction of Entities and Relations without Chain Dependency} (On going)\\
    The recent Seq2Seq model for joint extraction has innate drawbacks on complex classification. We analyze the reason why the F1 score decreasing with the decoding length increase. Then we propose two remedies: 1. bi-directional decoding. 2. multihead decoding.

    {\bf Long Tail Classification in Distant Supervised Relation Extraction} (On going)\\
    Long tail labels are notorious in DSRE. Inspired by the F-principle that neural networks prone to learn from high frequent labels to low frequent labels, we propose a hierarchical training hierarchical classification model to better capture the features in long tail relations.
    
% \\{\bf Microtunneling}\\
% Presented a seminar on Micro Tunneling, explaining its advantages over conventional method of drainage laying systems. Analysis considering direct and indirect cost of micro tunneling was also discussed.

\end{rSection}
%----------------------------------------------------------------------------------------
%	TECHNICAL STRENGTHS SECTION
%----------------------------------------------------------------------------------------

\begin{rSection}{Technical Strengths}

\begin{tabular}{ @{} >{\bfseries}l @{\hspace{6ex}} l }
Programming \ & Python, Haskell, PyTorch \\
Tools & shell, LaTeX \\
\end{tabular}

\end{rSection}

%----------------------------------------------------------------------------------------
%	WORK EXPERIENCE SECTION
%----------------------------------------------------------------------------------------

% \begin{rSection}{Work Experience}

% \begin{rSubsection}{SJ Contracts, Pune}{June 2016}{Site Engineer}{}
% \item On-site internship under this leading construction company. Learned and implemented various aspects such as quantity estimation, labour management and safety precautions.
% \end{rSubsection}


% \end{rSection}


% %	EXAMPLE SECTION
% %----------------------------------------------------------------------------------------

% \begin{rSection}{Academic Achievements} 
%  Runners up in B.G.Shirke Vidyarthi Competition for Innovative Project organized by Pune Construction Engineering Research Foundation in January 2018
% \item Won First Prize in Model Making Competition Organized by Symbiosis Institute of Technology, Pune.
% \end{rSection}

\begin{rSection}{Papers}
    D. Zeng*, \textbf{H. Zhang}*, Q. Liu, CopyMT: Multi-task Copy Mechanism for Joint Extraction of Entities and Relations (submitted)

    D. Zeng*, \textbf{H. Zhang}, J. Tian, Jointly Extracting Entities and Relations using Attention-based Multi-head Selection and BERT (submitted)

    D. Zeng*, \textbf{H. Zhang}*, L. Xiang, J. Wang and G. Ji, User-Oriented Paraphrase Generation With Keywords Controlled Network, in IEEE Access, vol. 7, pp. 80542-80551, 2019.     doi: 10.1109/ACCESS.2019.2923057
\end{rSection}
%----------------------------------------------------------------------------------------
% Extra Curricular
%----------------------------------------------------------------------------------------
% \begin{rSection}{Extra-Cirrucular} \itemsep -3pt
% \item Co-Organized “ Nirmitee 2017” - a National Symposium of Civil Department of MIT, Pune
% \item Attended a workshop on Autodesk Revit at IIT Bombay in 2014.
% \item Winner of Inter Departmental Football Competition 2015.
% \item Member of the  Rotaract Club Of Pune Pride from 2014 to 2017.
% \item Worked for a start-up company Named OUST as a Regional Marketing Manager
% %\item Trained and disciplined in National Cadet Corps (NCC), IIT Kanpur for a year.
%  %\item  Participated in Vijyoshi Camp 2012 organized at Indian Institute of Science, Bangalore.
%  %\item Won 2nd position in Kho-Kho in Intramurals conducted by Physical Education Section, IIT Kanpur.
%  %\item Pursued French as second language during secondary school from Grade 6 to Grade 10. Also participated in French Song Competition and French G.K. Quiz in Class 10th. %

% \end{rSection}

% \begin{rSection}{Personal Traits}
% \item Highly motivated and eager to learn new things.
% \item Strong motivational and leadership skills.
% \item Ability to work as an individual as well as in group.
% \end{rSection}
\newpage

\end{document}
