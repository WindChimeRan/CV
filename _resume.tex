% !TEX program = xelatex
% This is my resume
% by ice1000
\documentclass{resume}
%\usepackage{zh_CN-Adobefonts_external} % Simplified Chinese Support using external fonts (./fonts/zh_CN-Adobe/)
%\usepackage{zh_CN-Adobefonts_internal} % Simplified Chinese Support using system fonts

\usepackage{tabularx}

\begin{document}
\pagenumbering{gobble} % suppress displaying page number

\name{Mr. Haoran ZHANG}

\basicInfo{
  \email{haoranzh316@163.com} \textperiodcentered\
  \phone{0086-15674918015} \textperiodcentered\
  \github[WindChimeRan]{https://github.com/WindChimeRan}
  % \linkedin[billryan8]{https://www.linkedin.com/in/billryan8}
}

\section{\faGraduationCap\ Education}
\datedsubsection{\textbf{Changsha University of Science \& Technology (CSUST)}, Changsha, China}{2014.09 -- Present}
\datedsubsection{\textbf{Major: }Track Traffic Signal and Control \hfill{GPA: 2.76/4}}{09/2014 - 07/2015}
\datedsubsection{\textbf{Major: }Computer Science and Technology
  \hfill{GPA: 3.80/4 \ \ \ }}{09/2015 - 06/2018}

\datedsubsection{\textbf{Overall GPA}: 3.27/4, 83.21/100}{}
\datedsubsection{\textbf{WES GPA}: 3.34/4}{}
\datedsubsection{\textbf{Degree}: Bachelor of Engineering}{Expected in June 2018}
% \subsection{Overall GPA:}\hfill{3.27/4.0, 83.21/100}{}
% platform I am familiar with

\section{\faLaptop\ Skills}


\datedsubsection{\textbf{Deep Learning \& Machine Learning}}{}
 I learnt it from documentations, videos, papers and books, and practiced with GPUs in
 my lab.
 \begin{itemize}
  
 \item \textbf{TensorFlow}: 1 year of experience; \\
   {\url{https://github.com/WindChimeRan/Learn-Tensorflow-Notes}}
  \item \textbf{Python3}: 2 years of experience with focus on implementing
    machine learning algorithms;
  \item \textbf{Online Stanford CS231n Convolutional Neural Networks for Visual
      Recognition}: \\
  familiarized with CNNs,RNNs,GANs, and read over 60 papers; 
\item \textbf{Coursera Course}: \textit{Machine Learning}(Stanford University, 8 weeks)
\item \textbf{Coursera Course}: \textit{Probability Graph Models}(Stanford University, 2
   weeks, In-progress)
 \end{itemize}

 \datedsubsection{\textbf{Other Skills}}{}
 PyTorch, Cuda, OpenCV, LaTeX, Spacemacs, Linux, μC/OS-II; \\
 Haskell, scheme, C, C++, Assembly, Java, Matlab
 
 \datedsubsection{\textbf{Relavant Course Papers}}{}
 \begin{itemize}
   \item \textit{Compiler Principles and Technique A}: Mechanism and Compiling
     Optimization of TensorFlow
   \item \textit{Digital Image Processing}: K-means for Image Segmentation
   \item \textit{The Principle of Computer Organization A}: Difference in Programming
     with CPU and GPU
 \end{itemize}
\section{\faLightbulbO\ Research Experiences}
\datedsubsection{\textbf{Big Data and Machine Learning Project}}{10/2016 -
  06/2017}

An independent research program to build a deep neural network for satellite
imagery segmentation using variety of the-state-of-the-art techniques, especially Fully Convolutional Network, ResNeXt,
VGG19. The project finally compared the performance of different backbones with
and without transfer learning.
% \role{Kotlin, C\#, Racket, Ruby}{Founder and main contributer to the
% Kotlin/C\#/Ruby version}
\\ 
\textit{Advisor: Roy Yang}, Data Scientist \hfill{UCLA Academic Program}
\begin{itemize}
 \item Research Content: deep learning semantic segmentation for satellite
   imagery; \\
   {\url{https://github.com/WindChimeRan/Deep-Satellite}}
 \item Took responsibility for dataset making, algorithm building, data
visualization and report writing;
  \item Built dataset from DSTL, compared a transferred version and
    non-transferred version with the better one converged in 12 hours, and the
    mIoU was enhanced to 0.89.

\end{itemize}



\datedsubsection{\textbf{Unmanned Aerial Vehicle Design}}{11/2016 - 04/2017}
% \role{Kotlin, C\#, Racket, Ruby}{Founder and main contributer to the
% Kotlin/C\#/Ruby version}
\textit{Advisor: Prof. Zhou Shuren} \hfill{Dean of Computer Science, CSUST}
\begin{itemize}
 \item Research Content: Aerial Vehicle Camera Algorithm Design;
 \item Took responsibility for camera driver modifying, tracing algorithm building and
project management;
 \item Built a moving objects recognition algorithm with inter-frame difference method.


\end{itemize}

\datedsubsection{\textbf{Forecasting Traffic Flow Based on Genetic Neural Network}}{09/2015 - 04/2016}
\textit{Role: Group Leader}{}
\begin{itemize}
 \item Research Content: Regression algorithm Improvement;
 \item Responsibility: Collected data, programmed selection/crossover/mutation/back propagation/gradient checking.
module, made final presentation;
 \item Earned College Scientific Research Reward, CSUST.
\end{itemize}

\datedsubsection{\textbf{Course Project on C++ MFC Gobang Game \& AI Design}}{05/2016}
\textit{Role: Group Leader}{}
\begin{itemize}
 \item Research Content: GUI, LAN, and AI design;
 \item Responsibility: Programming includes GUI, LAN, and AI module;
 \item Got the highest score of 92 in the course \textit{Basic of the Software System Design Training} (C++
MFC) among the whole department of 256 students, ranking Top 1.
\end{itemize}

% Reference Test
%\datedsubsection{\textbf{Paper Title\cite{zaharia2012resilient}}}{May. 2015}
%An xxx optimized for xxx\cite{verma2015large}
%\begin{itemize}
%  \item main contribution
%\end{itemize}

\section{\faUniversity\ ACADEMIC ACTIVITIES}
\datedsubsection{\textbf{Machine Vision Lab}}{09/2016 - 07/2017}
\textit{Role: Co-founder \& Leader}{}
\begin{itemize}
 \item Co-founded the lab for the extended learning of machine vision for undergraduates and cooperated with graduates;
 \item Organized group seminar once a week between undergraduates and graduates, built an
Introduction of Machine Vision Repository(See Link); 
\\
\url{https://github.com/WindChimeRan/Machine-Vision-Lab-Tutorial/tree/master/Hello\%20Python}

 \item Conducted unmanned aerial vehicle project.
\end{itemize}

\datedsubsection{\textbf{Graduate Seminars of Computer Vision Track Team}}{09/2016 - 07/2017}
\textit{Role: Member}{}
\begin{itemize}
 \item Made presentations on TensorFlow tutorial with concentration on semantic
   segmentation, and
shared the experiments about deep learning I was engaged in.

\end{itemize}

\datedsubsection{\textbf{The Electrician Mathematical Contest in Modeling}}{09/2014 - 09/2017}
\textit{Role: Participant}{}
\begin{itemize}
 \item Participated in School-level Mathematical Contest in Modeling in 2014 and 2015, and attended
the National Electrician Mathematical Contest in Modeling in 2017;
\item Teamed with three members to solve one real problem within three days, in
  the areas of biostatistics,
optimization, and data engineering;
\item Responsible for coding and rapid prototyping;
\item Earned 3rd Prize in The Electrician Mathematical Contest in Modeling (National)

\end{itemize}

\datedsubsection{\textbf{GPU Technology Conference 2016 (GTC 2016)}}{09/2016}
\textit{Role: Participant}{}
\begin{itemize}
 \item Received one-day training on deep learning hosted by NVIDIA;
\item Learned about current development of AI and the most cutting-edge knowledge in deep
learning, for example, Caffe, TensorFlow, CNTK, LeNet, sliding window, FCN;
\item Earned Certificate of Attendance.
\end{itemize}


% \section{\faInfo\  Miscellaneous}
% \begin{itemize}[parsep=0.5ex]
%  \item Blog(Chinese): \url{http://blog.csdn.net/xsfl1234}
%  \item Languages: English - fluent (TOEFL 100), Chinese - native speaker
% \end{itemize}

\section{\faStar\ ACADEMIC HONORS}
\begin{enumerate}
 \datedline{\textit{\nth{2} class Scholarship for 2016-2017 Academic Year}, CSUST}{09/2017}

 \datedline{\textit{\nth{3} class Scholarship for 2015-2016 Academic Year}, CSUST}{09/2016}
 \datedline{\textit{\nth{3} Prize in ACM-CSUST Collegiate Programming Contest}}{07/2016}
 \datedline{\textit{College Scientific Research Reward}, CSUST}{02, 04/2016}
 \datedline{\textit{Scholarship for Academic Competition Excellence},
   CSUST}{09/2015}
 \datedline{\textit{\nth{2} Prize in ACM-CSUST Collegiate Programming Contest}}{07/2015}
 \datedline{\textit{\nth{3} Prize in The Electrician Mathematical Contest in
     Modeling}, National}{05/2015}
\end{enumerate}
%% Reference
%\newpage
%\bibliographystyle{IEEETran}
%\bibliography{mycite}
\end{document}

